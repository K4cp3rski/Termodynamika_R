\documentclass[12pt,a4paper]{article}
\usepackage[utf8]{inputenc}
\usepackage[T1]{fontenc}
\usepackage{fontspec}
\usepackage[polish]{babel}
\usepackage{amsmath}
\usepackage{graphicx}
\usepackage[table,xcdraw]{xcolor}
\usepackage{hhline}
\usepackage{placeins}
\usepackage[margin=0.6in]{geometry}
\usepackage{appendix}
\usepackage{colortbl}
\usepackage{physics}
\usepackage{float}
\usepackage{datetime}
\usepackage{hyperref}

\title{Praca Domowa Termodynamika i Fizyka Statystyczna R 2021/2022}
\author{Kacper Cybiński}
% \newdate{date}{28}{01}{2022}
% \date{\displaydate{date}}
\date{\today}
\setlength\parindent{0pt}

\newcommand{\com}[1]{{\color{red} #1}}

\newcommand{\link}[2]{{\color{cyan} \href{#1}{#2}}}

\renewcommand{\emph}{\textbf}

\begin{document}

\maketitle

\section{Zadanie 3}

Ekwipartycja energii. Klasyczny uklad fizyczny pozostaje $w$ kontakcie z termostatem o temperature $T$. Wiemy, że energia potencjalna tego układu dąży do $\infty$ na brzegach (ściankach) układu. Pokaż, że
(a) Średnia energia kinetyczna na stopień swobody jest równa $\frac{1}{2} k T$
(b) zachodzi relacja
$$
\left\langle q_{i} \frac{\partial V}{\partial q_{j}}\right\rangle=k T \delta_{i j}
$$


\section{Rozwiązanie}

Rozwaźmy hamiltonian układu postaci:
$$
H=\sum_{i=0}^{3 N} \frac{p_{i}^{2}}{2 m}+V\left(q_{1}, q_{2}, \ldots, q_{N}\right)
$$
Suma statystyczna dana jest wzorem:
$$
Z=\int d \Gamma_{S} \exp (-\beta H)=\frac{1}{N ! h^{3 N}} \int d^{3 N} p d^{3 N} q \exp (-\beta H)
$$
gdzie $\beta=\frac{1}{k T}$ i $T$ jest temperaturą termostatu. Spróbujmy wyznaczyć teraz średnią wartość energii kinetycznej na stopień swobodny. Ma ona postać:
$$
\left\langle E_{k i n}\right\rangle=\frac{\int d p_{i} \frac{p_{1}^{2}}{2 m} \exp \left(-\beta \frac{p_{1}^{2}}{2 m}\right) A}{\int d p_{i} \exp \left(-\beta \frac{p_{1}^{2}}{2 m}\right) A}
$$
gdzie $A$ to całki po przestrzeni fazowej bez całki po $p_{i}$. Stąd:
$$
\left\langle E_{k i n}\right\rangle=\frac{\int d p_{i} \frac{p_{1}^{2}}{2 m} \exp \left(-\beta \frac{p_{i}^{2}}{2 m}\right)}{\int d p_{i} \exp \left(-\beta \frac{p_{1}^{2}}{2 m}\right)}
$$
Zauważmy, że całka w liczniku jest pochodną po $\beta$ całki z mianownika z dodanym minusem:
$$
\left\langle E_{k i n}\right\rangle=-\frac{1}{\int d p_{i} \exp \left(-\beta \frac{p_{i}^{2}}{2 m}\right)} \frac{\partial \int d p_{i} \exp \left(-\beta \frac{p_{i}^{2}}{2 m}\right)}{\partial \beta}
$$
Całka $\int d p_{i} \exp \left(-\beta \frac{p_{i}^{2}}{2 m}\right)$ jest całką gaussowską, więc wynik całki po wszystkich wartości $p_{i}$ od $-\infty$ do $\infty$ daje wynik:
$$
\int d p_{i} \exp \left(-\beta \frac{p_{i}^{2}}{2 m}\right)=\sqrt{\frac{2 m \pi}{\beta}}
$$
A więc średnia wartość energii kinetycznej na stopień swobody jest równa:
$$
\left\langle E_{\text {kin }}\right\rangle=-\frac{1}{\sqrt{\frac{2 m \pi}{\beta}}} \frac{\partial \sqrt{\frac{2 m \pi}{\beta}}}{\partial \beta}=\frac{1}{2 \beta}=\frac{1}{2} k T
$$
A zatem średnia wartość energii kinetycznej na stopień swobody jest równa $\frac{1}{2} k T$ co należało udowodnić.
Zajmijmy się teraz drugą częścią zadania. Należy jeszcze udowodnić, że zachodzi:
$$
\left\langle q_{i} \frac{\partial V}{\partial q_{j}}\right\rangle=\delta_{i j} k T
$$

Rozpatrzmy osobno przypadek gdy $i=j$ i $i \neq j .$ Dla $i=j$ mamy:
$$
\frac{1}{Z} \int d \Gamma_{S} \exp (-\beta H)=1
$$
Całkując lewą stronę przez części dla dowolnego $i$ dostajemy:
$$
L H S=\left.\frac{1}{Z N ! h^{3 N}} q_{i} \int d^{3 N} p_{i} d^{3 N-1} q \exp (-\beta H)\right|_{x_{0}^{\prime}} ^{x_{0}}+\frac{1}{Z N ! h^{3 N}} \int d q_{i} q_{i} \beta \frac{\partial V}{\partial q_{i}} \int d^{3 N} p_{i} d^{3 N-1} q \exp (-\beta H)
$$
gdzie $x_{0}$ i $x_{0}^{\prime}$ to granice całkowania dla $q_{i}$ odpowiadajace brzegom układu. Wiadomo, że na brzegach potencjał dąży do $\infty$, więc człony brzegowe są równe 0 . Stąd w wyrażeniu zostaje tylko druga całka. Zauważmy, że jest ona równa $\beta\left\langle q_{i} \frac{\partial \mathcal{V} q_{j}}{}\right\rangle . \mathrm{Z}$ drugiej strony jest ona równa 1. Stąd mamy, że:
$$
\left\langle q_{i} \frac{\partial V}{\partial q_{i}}\right\rangle=\frac{1}{\beta}=k T
$$
Weźmy teraz przypadek $i \neq j$. Mamy:
$$
\left\langle q_{i} \frac{\partial V}{\partial q_{j}}\right\rangle=\frac{1}{Z} \int d \Gamma_{S} q_{i} \frac{\partial V}{\partial q_{j}} \exp (-\beta H)
$$
Przecałkujmy prawą stronę przez części, tym razem z względu na $q_{j}$ :
$$
R H S=-\left.\frac{1}{Z N ! h^{3 N} \beta} \int d^{3 N} p d^{3 N-1} q q_{i} \exp (-\beta H)\right|_{x_{0}^{x_{0}}}+\frac{1}{\beta Z N ! h^{3} N} \int d^{3 N} p d^{3 N} \frac{\partial q_{i}}{\partial q_{j}} \exp (-\beta H)
$$
Dla wyrazów brzegowych eksponens ponownie dąży do zera, więc pierwszy człon jest równy 0 . W drugim członie pod całką znajduje się wyrażenie $\frac{\partial q_{i}}{\partial q j}$, które ze względu na to, że współrzędne uogólnione są od siebie niezależne, jest równe 0 . Czyli dla $i \neq j$ :
$$
\left\langle q_{i} \frac{\partial V}{\partial q_{i}}\right\rangle=0
$$
Składając oba przypadki dostajemy wzór, który należało udowodnić:
$$
\left\langle q_{i} \frac{\partial V}{\partial q_{j}}\right\rangle=\delta_{i j} k T
$$
\end{document}