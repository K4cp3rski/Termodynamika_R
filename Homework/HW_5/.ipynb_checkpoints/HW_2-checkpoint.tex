\documentclass[12pt,a4paper]{article}
\usepackage[utf8]{inputenc}
\usepackage[T1]{fontenc}
\usepackage{fontspec}
\usepackage[polish]{babel}
\usepackage{amsmath}
\usepackage{graphicx}
\usepackage[table,xcdraw]{xcolor}
\usepackage{hhline}
\usepackage{placeins}
\usepackage[margin=0.6in]{geometry}
\usepackage{appendix}
\usepackage{colortbl}
\usepackage{physics}
\usepackage{float}
\usepackage{datetime}
\usepackage{hyperref}

\title{Praca Domowa Termodynamika i Fizyka Statystyczna R 2021/2022}
\author{Kacper Cybiński}
% \newdate{date}{28}{01}{2022}
% \date{\displaydate{date}}
\date{\today}
\setlength\parindent{0pt}

\newcommand{\com}[1]{{\color{red} #1}}

\newcommand{\link}[2]{{\color{cyan} \href{#1}{#2}}}

\renewcommand{\emph}{\textbf}

\begin{document}

\maketitle

\section{Zadanie 2}

Model suwakowy DNA. Suwak składa się $N$ par ząbków, z których każda może być połączona lub rozłączona. Energia połączonej pary wynosi 0 , a rozłączonej $E$. Ponadto wymagamy, aby suwak otwierał się tylko z jednej strony (na przykład z lewej) - a zatem para $k$ może być otwarta tylko wtedy, gdy otwarte są pary $1,2, \ldots, k-1$ (patrz rys.). Znajdź sumę statystyczną dla takiego układu oraz wyznacz średnią liczbę otwartych par ząbków $<N_{o t w}>w$ temperaturze $T$. Pokaż, że dla malych temperatur (tj. $k T \ll E$ ) średnia liczba otwartych par ząbków $<N_{o t w}>$ jest niezależna od $N$. Jaką wartość przybiera $<N_{o t w}>$ w wysokich temperaturach (tj. dla $k T \gg E)$ ?


\section{Rozwiązanie}

Rozważmy sytuację, gdy rozłączone jest pierwsze n par ząbków. Skoro energia rozłączonej pary jest równa $\mathrm{E}$, a polączonej 0 , to energia układu w takim stanie równa jest $E_{n}=n E$. Suma statystyczna będzie sumą dla wszystkich takich stanów energii i dana będzie wzorem:
$$
Z=\sum_{n=0}^{N} \exp (-\beta E n)=\frac{\exp (-\beta E(N+1))-1}{\exp (-\beta E)-1}
$$
Prawdopodobieństwo znalezienia się układu w stanie o energii $E_{n}$ równe jest:
$$
p_{n}=\frac{\exp (-\beta E n)}{Z}
$$
A stąd średnia liczba otwartych par jest postaci:
$$
\left\langle N_{o t w}\right\rangle=\sum_{n=0}^{N} n p_{n}=\frac{1}{Z} s u m_{n=0}^{N} n \exp (-\beta E n)=-\frac{1}{Z}\left(\frac{\partial \exp (-\beta E n)}{\partial E}\right)_{T}=-\left(\frac{\partial \log Z}{\partial E}\right)_{T}
$$
Rozpiszmy wyrażenie $\log Z$ :
$$
\log Z=\log \left(\frac{\exp (-\beta E(N+1))-1}{\exp (-\beta E)-1}\right)=\log (\exp (-\beta E(N+1))-1)-\log (\exp (-\beta E)-1)
$$
Wracając z podstawieniem do wyrażenia na średnią liczbę otwartych par ząbków:
$$
\begin{gathered}
\left\langle N_{o t w}\right\rangle=\frac{\beta(N+1) \exp (-\beta E(N+1))}{\exp (-\beta E(N+1))-1}-\frac{\beta \exp (-\beta E)}{\exp (-\beta E)-1} \\
\left\langle N_{\text {otw }}\right\rangle=\frac{\beta(N+1) \exp (-\beta E(N+1))(\exp (-\beta E)-1)-\beta \exp (-\beta E)(\exp (-\beta E(N+1))-1)}{(\exp (-\beta E(N+1))-1)(\exp (-\beta E)-1)}
\end{gathered}
$$
Rozważmy teraz granicę niskich temperatur $k T \ll E$. W tej granicy wyrażenie $\beta E \rightarrow \infty$, więc $\exp (-\beta E) \rightarrow 0$. Wynika stąd, że średnia liczba otwartych par ząbków będzie równa po prostu:
$$
\left\langle N_{o t w}\right\rangle=0
$$
Co jak należało udowodnić jest niezależne od całkowitej liczby par ząbków.
Rozważmy jeszcze granice dla $k T \gg E$. Wówczas wyrażenie $\beta E$ dąży do 0 , a więc $\exp (-\beta E) \rightarrow 1$. Wprowadźmy oznaczenie $x=\exp (-\beta E)$. Policzmy teraz przejście graniczne dla $x \rightarrow 1$ :
$$
\left\langle N_{o t w}\right\rangle=\frac{N}{2}
$$
\end{document}