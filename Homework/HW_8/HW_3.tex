\documentclass[12pt,a4paper]{article}
\usepackage[utf8]{inputenc}
\usepackage[T1]{fontenc}
\usepackage{fontspec}
\usepackage[polish]{babel}
\usepackage{amsmath}
\usepackage{graphicx}
\usepackage[table,xcdraw]{xcolor}
\usepackage{hhline}
\usepackage{placeins}
\usepackage[margin=0.6in]{geometry}
\usepackage{appendix}
\usepackage{colortbl}
\usepackage{physics}
\usepackage{float}
\usepackage{datetime}
\usepackage{hyperref}

\title{Praca Domowa Termodynamika i Fizyka Statystyczna R 2021/2022}
\author{Kacper Cybiński}
% \newdate{date}{28}{01}{2022}
% \date{\displaydate{date}}
\date{\today}
\setlength\parindent{0pt}

\newcommand{\com}[1]{{\color{red} #1}}

\newcommand{\link}[2]{{\color{cyan} \href{#1}{#2}}}

\renewcommand{\emph}{\textbf}

\begin{document}

\maketitle

\section{Zadanie 3}

N moli gazu doskonałego poddano przemianie ze stanu początkowego opisanego parametrami $T_{1}, V_{1}$ do stanu końcowego o objętości $V_{2}$, przy czym $V_{2}>V_{1}$. Rozważyć dwie przemiany:
a) $p(V)=\gamma-\alpha\left(V-V_{1}\right)$
b) $p(V)=\gamma-\beta\left(V-V_{1}\right)^{2}$
Współczynniki $\alpha, \beta, \gamma$ dobrano tak, aby ciśnienie końcowe w obydwu przemianach było jednakowe. Wyznacz zależność $T(V)$, temperaturę końcową $T_{2}$ oraz pracę wykonaną przez siły zewnętrzne w obydwu przypadkach. Wyznaczyć warunek aby było spełnione: $T_{2}>T_{1}$.

\section{Rozwiązanie}

Rozważmy pierwszą przemianę, dla której $p(V)$ jest dane wzorem:
$$
p(V)=\gamma-\alpha\left(V-V_{1}\right)
$$
Jako że pracujemy z gazem doskonałym, to spełnione jest także równanie Clapeyrona:
$$
p V=N R T
$$
Przekształcając je i podstawiając zależność $p(V)$ dostaniemy zależność $T(V)$ :
$$
T(V)=\frac{V\left(\gamma-\alpha\left(V-V_{1}\right)\right)}{N R}
$$
A stąd temperatura końcowa jest równa:
$$
T_{2}=T\left(V_{2}\right)=\frac{V_{2}\left(\gamma-\alpha\left(V_{2}-V_{1}\right)\right)}{N R}
$$
Natomiast praca wykonana przez siły zewnętrzne jest równa:
$$
W=-\int_{V_{1}}^{V_{2}} p d V=-\int_{V_{1}}^{V_{2}}\left(\gamma-\alpha\left(V-V_{1}\right)\right) d V=-\gamma\left(V_{2}-V_{1}\right)+\frac{\alpha}{2}\left(V_{2}-V_{1}\right)^{2}
$$
Wyznaczmy dla tej przemiany warunek na $T_{2}>T_{1}$ :
$$
\begin{gathered}
T\left(V_{2}\right)>T\left(V_{1}\right) \\
\frac{V_{2}\left(\gamma-\alpha\left(V_{2}-V_{1}\right)\right)}{N R}>\frac{V_{1} \gamma}{N R} \\
\gamma\left(V_{2}-V_{1}\right)-\alpha V_{2}\left(V_{2}-V_{1}\right)>0
\end{gathered}
$$
Wiadomo, że $V_{2}>V_{1}$, czyli wynika z tego, że warunek $T_{2}>T_{1}$ będzie spelniony tylko wtedy, gdy ${ }_{\alpha}^{\gamma}>V_{2}$. Rozważmy drugą przemianę, gdzie $p(V)$ jest dane:
$$
p(V)=\gamma-\beta\left(V-V_{1}\right)^{2}
$$
Postępując analogicznie dostajemy, że $T(V)$ jest dane wzorem:
$$
T(V)=\frac{V\left(\gamma-\beta\left(V-V_{1}\right)^{2}\right)}{N R}
$$
A stąd $T_{2}$ :
$$
T_{2}=T\left(V_{2}\right)=\frac{V_{2}\left(\gamma-\beta\left(V_{2}-V_{1}\right)^{2}\right)}{N R}
$$
Natomiast praca wykonana przez siły zewnętrzne jest równa:
$$
W=-\int_{V_{1}}^{V_{2}} p d V=-\gamma\left(V_{2}-V_{1}\right)+\frac{\beta}{3}\left(V_{2}-V_{1}\right)^{3}
$$

Sprawdźmy na koniec kiedy będzie spełniony warunek $T_{2}>T_{1}$ :
$$
\begin{gathered}
T\left(V_{2}\right)>T\left(V_{1}\right) \\
\frac{V_{2} \gamma-\beta V_{2}\left(V_{2}-V_{1}\right)^{2}}{N R}>\frac{V_{1} \gamma}{N R} \\
\left(V_{2}-V_{1}\right)\left(\gamma-\beta V_{2}\left(V_{2}-V_{1}\right)\right)>0
\end{gathered}
$$
Czyli musi być spełnione:
$$
\gamma-\beta V_{2}\left(V_{2}-V_{1}\right)>0
$$
Zauważmy, że stałe $\alpha, \beta, \gamma$ dobrano tak, by ciśnienie końcowe w obu przemianach bylo takie samo, to znaczy, żeby zachodziła równość:
$$
\gamma-\alpha\left(V_{2}-V_{1}\right)=\gamma-\beta\left(V_{2}-V_{1}\right)^{2}
$$
A stąd mamy związek $\alpha$ z $\beta$ :
$$
\alpha=\beta\left(V_{2}-V_{1}\right)
$$
Czyli warunek dla drugiej przemiany ponownie redukuje się do nierówności $\frac{\gamma}{\alpha}>V_{2}$. Dodatkowo, zauważmy, że $p_{1}=\gamma$ dla obu przemian. Czyli ostatecznie żeby $T_{2}>T_{1}$ dla obu przemian, to stałe muszą spełniać warunki:
$$
\begin{gathered}
\alpha<\frac{p_{1}}{V_{2}} \\
\beta<\frac{p_{1}}{V_{2}\left(V_{2}-V_{1}\right)}
\end{gathered}
$$

\end{document}