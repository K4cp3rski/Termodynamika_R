\documentclass[12pt,a4paper]{article}
\usepackage[utf8]{inputenc}
\usepackage[T1]{fontenc}
\usepackage{fontspec}
\usepackage[polish]{babel}
\usepackage{amsmath}
\usepackage{graphicx}
\usepackage[table,xcdraw]{xcolor}
\usepackage{hhline}
\usepackage{placeins}
\usepackage[margin=0.6in]{geometry}
\usepackage{appendix}
\usepackage{colortbl}
\usepackage{physics}
\usepackage{float}
\usepackage{datetime}
\usepackage{hyperref}

\title{Praca Domowa Termodynamika i Fizyka Statystyczna R 2021/2022}
\author{Kacper Cybiński}
% \newdate{date}{28}{01}{2022}
% \date{\displaydate{date}}
\date{\today}
\setlength\parindent{0pt}

\newcommand{\com}[1]{{\color{red} #1}}

\newcommand{\link}[2]{{\color{cyan} \href{#1}{#2}}}

\renewcommand{\emph}{\textbf}

\begin{document}

\maketitle

\section{Zadanie 2}

Jeden mol gazu doskonałego o znanej wartości $C_{V}$ przeszedł ze stanu opisanego parametrami $p_{0}, V_{0}$ do stanu o objętości $V_{1}=2 V_{0}$. Przemiana była prowadzona z zachowaniem równości $p^{2} V=$ const. Wyznacz wykonaną pracę, ciepło wymienione z otoczeniem oraz zmianę energii wewnętrznej gazu.

\section{Rozwiązanie}

Praca wykonana w trakcie przemiany jest dana wzorem:
$$
W=-\int_{V_{0}}^{V_{1}} p d V
$$
W trakcie przemiany zachowana jest wartość $p^{2} V$, a więc także wartość $p \sqrt{V}$. Stąd można napisać:
$$
\begin{gathered}
W=-\int_{V_{0}}^{V_{1}} p \sqrt{V} \frac{d V}{\sqrt{V}}=-\int_{V_{0}}^{V_{1}} p_{0} \sqrt{V_{0}} \frac{d V}{\sqrt{V}} \\
W=-2 p_{0} \sqrt{V_{0}}\left(\sqrt{V_{1}}-\sqrt{V_{0}}\right)=-2 p_{0} V_{0}(\sqrt{2}-1)
\end{gathered}
$$
Policzmy teraz zmianę energii wewnętrznej:
$$
\Delta U=\int_{T_{0}}^{T_{1}} c_{V} d T=c_{V}\left(T_{1}-T_{0}\right)=\frac{C_{V}}{R}\left(2 p_{1} V_{0}-p_{0} V_{0}\right)
$$
Wiadomo że $p_{1}^{2} V_{1}=p_{0}^{2} V_{0}$, a stąd dostać można $p_{1}=\frac{1}{\sqrt{2}} p_{0}$. Podstawiając tę zależność do wzoru na zmianę energii wewnętrznej dostajemy:
$$
\Delta U=\frac{C_{V}}{R} p_{0} V_{0}(\sqrt{2}-1)
$$
Dostarczone ciepło wyznaczyć można z pierwszej zasady termodynamiki:
$$
Q=\Delta U-W=\frac{C_{V}}{R} p_{0} V_{0}(\sqrt{2}-1)+2 p_{0} V_{0}(\sqrt{2}-1)=p_{0} V_{0}(\sqrt{2}-1)\left(\frac{C_{V}+2 R}{R}\right)
$$

\end{document}