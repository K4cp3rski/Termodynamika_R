\documentclass[12pt,a4paper]{article}
\usepackage[utf8]{inputenc}
\usepackage[T1]{fontenc}
\usepackage{fontspec}
\usepackage[polish]{babel}
\usepackage{amsmath}
\usepackage{graphicx}
\usepackage[table,xcdraw]{xcolor}
\usepackage{hhline}
\usepackage{placeins}
\usepackage[margin=0.6in]{geometry}
\usepackage{appendix}
\usepackage{colortbl}
\usepackage{physics}
\usepackage{float}
\usepackage{datetime}
\usepackage{hyperref}

\title{Praca Domowa Termodynamika i Fizyka Statystyczna R 2021/2022}
\author{Kacper Cybiński}
% \newdate{date}{28}{01}{2022}
% \date{\displaydate{date}}
\date{\today}
\setlength\parindent{0pt}

\newcommand{\com}[1]{{\color{red} #1}}

\newcommand{\link}[2]{{\color{cyan} \href{#1}{#2}}}

\renewcommand{\emph}{\textbf}

\begin{document}

\maketitle

\section{Zadanie 1}

Cykl Joule'a składa się z dwóch przemian izobarycznych przy ciśnieniu $p_1$ oraz  $p_2$, gdzie $(p_2 > p_1)$ oraz dwóch przemian adiabatycznych. Substancją roboczą jest gaz doskonały o znanych wartościach $C_v, C_p$ oraz $\gamma = C_p/C_v$. Oblicz sprawność silnika działającego w takim cyklu.
\section{Rozwiązanie}

Zauważmy najpierw, że zgodnie z definicją przemiany adiabatycznej nie jest w niej wymieniane cieplo z otoczeniem, stąd też jakakolwiek wymiana ciepla nastąpi tylko w przemianach izobarycznych. Dodatkowo bez straty ogólnosci zalożyć można, że ciepło pobrane przy rozprężaniu izobarycznym dla temperatury $p_{2}$ jest równe:
$$
Q_{1}=\int_{T_{1}}^{T_{2}} c_{V} d T+\int_{V_{1}}^{V_{2}} p_{2} d V=c_{V}\left(T_{2}-T_{1}\right)+p_{2}\left(V_{2}-V_{1}\right)=c_{p}\left(T_{2}-T_{1}\right)
$$
gdzie $c_{v}=C_{V} n$ i $c_{p}=C_{p} n$, a $T_{1}$ i $T_{2}$ to temperatury w odpowiednich punktach cyklu. Analogicznie dla sprężania mamy:
$$
Q_{2}=c_{p}\left(T_{4}-T_{3}\right)
$$
Praca wykonana w tym cyklu jest równa różnicy ciepła pobranego i oddanego, czyli:
$$
W=Q_{1}-Q_{2}=c_{p}\left(T_{2}-T_{1}-T_{4}+T_{3}\right)
$$
Stąd sprawność cyklu jest rowna:
$$
\eta=\frac{W}{Q_{1}}=1-\frac{T_{4}-T_{3}}{T_{2}-T_{1}}
$$
Zauważmy że przejścia z temperatury $T_{2}$ do $T_{3}$ i $T_{4}$ do $T_{1}$ nastepowały w trakcie przemiany adiabatycznej. Na podstawie zależności dla przemiany adiabatycznej postaci $p^{\frac{7-1}{y}} V=$ const wyznaczyć można temperatury w zależnosci od znanych wartości ciśnienia. Mamy stąd:
$$
\begin{aligned}
&T_{4}=\left(\frac{p_{2}}{p_{1}}\right)^{\frac{7-1}{\gamma}} T_{1} \\
&T_{3}=\left(\frac{p_{2}}{p_{1}}\right)^{\frac{p-1}{\gamma}} T_{2}
\end{aligned}
$$
Podstawiając do wzoru na sprawność dostajemy:
$$
\eta=1-\left(\frac{p_{2}}{p_{1}}\right)^{\frac{\gamma-1}{\gamma}}
$$


\end{document}