\documentclass[12pt,a4paper]{article}
\usepackage[utf8]{inputenc}
\usepackage[T1]{fontenc}
\usepackage{fontspec}
\usepackage[polish]{babel}
\usepackage{amsmath}
\usepackage{graphicx}
\usepackage[table,xcdraw]{xcolor}
\usepackage{hhline}
\usepackage{placeins}
\usepackage[margin=0.6in]{geometry}
\usepackage{appendix}
\usepackage{colortbl}
\usepackage{physics}
\usepackage{float}
\usepackage{datetime}
\usepackage{hyperref}

\title{Praca Domowa Termodynamika i Fizyka Statystyczna R 2021/2022}
\author{Kacper Cybiński}
% \newdate{date}{28}{01}{2022}
% \date{\displaydate{date}}
\date{\today}
\setlength\parindent{0pt}

\newcommand{\com}[1]{{\color{red} #1}}

\newcommand{\link}[2]{{\color{cyan} \href{#1}{#2}}}

\renewcommand{\emph}{\textbf}

\begin{document}

\maketitle

\section{Zadanie 3}

\emph{Debye again...} Oblicz energię wewnętrzną ciała stałego (opisywanego modelem Debye'a) w dla temperatury dążącej do zera i wyraź tę granicę przez temperaturę Debye'a. Pokaż że:
$$
\int_{0}^{\infty} C_{V}(T=\infty)-C_{V}(T)=U(T=0)
$$

\section{Rozwiązanie}

Wiemy, że energia wewnętrzna w modelu Debeye'a jest dana wzorem:
$$
U=\int_{0}^{\infty} \hbar \omega\left(\frac{1}{2}+\frac{1}{\exp (\beta \hbar \omega)-1}\right) g(\omega) d \omega
$$
Chcemy pokazać, że zachodzi taka równość:
$$
\int_{0}^{\infty}\left(C_{V}(T \rightarrow \infty)-C_{V}(T)\right) d T=U_{0}
$$
Tutaj przypominamy wiedzę z Wykładu, że dla $T \rightarrow \infty$ ciepło właściwe ma postać 
$
C_{V}(T \rightarrow \infty)=3 N k
$ i spełnia zależność różniczkową 
$
C_{V}=\left(\frac{\partial U}{\partial T}\right)_{V}
$.
W ogólności energia wewnętrzna jest dana wzorem:
$$
U=9 N k T\left(\frac{T}{T_{D}}\right)^{3} \int_{0}^{\frac{T_{D}}{T}} \frac{x^{3}}{\exp (x)-1} d x
$$
Widzimy, że jest ona funkcją tylko temperatury, więc pochodne cząstkowe przy wzorze na ciepło właściwe możemy zamienić na pełne pochodne. Upraszcza nam to naszą całkę do postaci:
$$
\int_{0}^{\infty} C_{V}(T) d T=\int_{0}^{\infty} \frac{\mathrm{d} U}{\mathrm{~d} T} d T=\int_{0}^{\infty} d U
$$

Czyli propagując to uproszczenie dalej mamy:
$$
\int_{0}^{\infty}\left(C_{V}(T \rightarrow \infty)-C_{V}(T)\right) d T=\left.3 N k T\right|_{0} ^{\infty}-\left.U(T)\right|_{0} ^{\infty}
$$

W tym miejscu warto by się było zastanowić co się dzieje z $U$ gdy $T \to 0$, zachowa się to trochę inaczej niż zwykle to obserwowaliśmy na ćwiczeniach, bo nie będzie można pominąć członu stałego.
Dzieje się tak, ponieważ zauważyć można, że człon pod całką, który zależy od temperatury dla $T \rightarrow 0$ także dąży do zera, czyli pozostaje wyraz stały. Stąd też energia w zerowej temperaturze będzie miała postać:

$$
\begin{gathered}
U_{0}=\int_{0}^{\omega_{D}} \hbar \omega \frac{1}{2} \frac{3 V}{2 \pi^{2} v_{0}^{3}} \omega^{2} d \omega \\
U_{0}=\frac{3 V \hbar}{16 \pi^{2} v_{0}^{3}} \omega_{D}^{4}
\end{gathered}
$$
Pamiętamy, że $\omega_{D}=v_{0}^{3} 6 \pi^{2} \frac{N}{V}$ a $T_{D}=\frac{\hbar \omega_{D}}{k}$. Po skorzystaniu z tej wiedzy wynik nam się upraszcza do postaci:
$$
U_{0}=\frac{9}{8} N k T_{D}
$$

Wróćmy teraz do pierwotnego zagadnienia.

Dla dużych temperatur $\mathrm{U}(\mathrm{T})$ przyjmuje postać $U(T)=3 N k T$, a dla $T \rightarrow 0$ jest to po prostu $U_{0}$, więc możemy zapisać:
$$
\lim _{T \rightarrow \infty} 3 N k T-\left(\lim _{T \rightarrow \infty} 3 N k T-U_{0}\right)=U_{0}
$$
Pokazaliśmy więc, że
$$
\int_{0}^{\infty}\left(C_{V}(T \rightarrow \infty)-C_{V}(T)\right) d T=U_{0}
$$

\end{document}