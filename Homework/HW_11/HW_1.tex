\documentclass[12pt,a4paper]{article}
\usepackage[utf8]{inputenc}
\usepackage[T1]{fontenc}
\usepackage{fontspec}
\usepackage[polish]{babel}
\usepackage{amsmath}
\usepackage{graphicx}
\usepackage[table,xcdraw]{xcolor}
\usepackage{hhline}
\usepackage{placeins}
\usepackage[margin=0.6in]{geometry}
\usepackage{appendix}
\usepackage{colortbl}
\usepackage{physics}
\usepackage{float}
\usepackage{datetime}
\usepackage{hyperref}

\title{Praca Domowa Termodynamika i Fizyka Statystyczna R 2021/2022}
\author{Kacper Cybiński}
% \newdate{date}{28}{01}{2022}
% \date{\displaydate{date}}
\date{\today}
\setlength\parindent{0pt}

\newcommand{\com}[1]{{\color{red} #1}}

\newcommand{\link}[2]{{\color{cyan} \href{#1}{#2}}}

\renewcommand{\emph}{\textbf}

\begin{document}

\maketitle

\section{Zadanie 1}

\emph{Fotony w d-wymiarach}. Oblicz energię wewnętrzną promieniowania we wnęce dla wnęki jedno- i dwu-wymiarowej

\section{Rozwiązanie}

Energia dla konkretnej częstości fotonu dana jest wzorem (fakt z ćwiczeń):
$$
E(\omega)=\frac{\hbar \omega}{\exp (\beta \hbar \omega)-1}
$$
Jak chcemy znaleźć energię wewnętrzną, to musimy do tego mieć gęstość stanów uzależnioną od częstości. W najprostszym, $1D$ przypadku, liczba stanów o $n \leq m$, gdzie $m$ i $n$ to numer modu, jest równa: $$N(m)=2 m$$
Wiadomo, że energię można powiązać z numerem modu 
$
\epsilon=\frac{2 \pi \hbar n}{L}
$\\
gdzie L to wymiar wnęki. Dodatkowo wypada tu przejść na rozmowę w języku częstości poprzez zastosowanie związku 
$
\epsilon=\frac{\hbar \omega}{c}
$
Teraz dostaliśmy zależność:
$$
n=\frac{\omega L}{2 \pi c}
$$
A stąd można wyznaczyć liczbę stanów $N(\omega)$ o częstości mniejszej niż $\omega$ i gęstość stanów $g(\omega)$uzyskaną poprzez różniczkowanie liczby stanów:
$$
N(\omega)=\frac{\omega L}{\pi c} \qc g(\omega)=\frac{L}{\pi c}
$$
Stąd energia wewnętrzna fotonów jest równa:
$$
\begin{gathered}
E=\int_{0}^{\infty} g(\omega) \frac{\hbar \omega}{\exp (\beta \hbar \omega)-1} d \omega \\
E=\frac{L}{\pi c} \int_{0}^{\infty} \frac{\hbar \omega}{\exp (\beta \hbar \omega)-1} d \omega \\
E=\frac{L k^{2} \pi}{6 \hbar c} T^{2}
\end{gathered}
$$
Przejdźmy do układu dwuwymiarowego. Liczba stanów o $n$ mniejszym od jakiegoś $m$ jest równa:
$$
N(m)=2 \pi|\vec{n}|^{2}
$$

Odpowiednio liczba stanów o częstości mniejszej od $\omega$, oraz gęstość stanów:
$$
N(\omega)=\frac{L^{2} \omega^{2}}{2 \pi^{2} c^{2}} \qc g(\omega)=\frac{L^{2}}{\pi^{2} c^{2}} \omega
$$
Czyli energia wewnętrzna gazu fotonów:
$$
E=\frac{L^{2}}{\pi^{2} c^{2}} \int_{0}^{\infty} \frac{\hbar \omega^{2}}{\exp (\beta \hbar \omega)-1} d \omega
$$
Wycałkowanie tego wyrażenia da nam postać:
$$
E=\frac{2 L^{2} k^{3}}{\hbar^{2} c^{2} \pi^{2}} T^{3} \zeta(3)
$$

\end{document}