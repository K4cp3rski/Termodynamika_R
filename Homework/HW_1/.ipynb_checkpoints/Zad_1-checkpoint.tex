\documentclass[12pt,a4paper]{article}
\usepackage[utf8]{inputenc}
\usepackage[T1]{fontenc}
\usepackage{fontspec}
\usepackage[polish]{babel}
\usepackage{amsmath}
\usepackage{graphicx}
\usepackage[table,xcdraw]{xcolor}
\usepackage{hhline}
\usepackage{placeins}
\usepackage[margin=0.6in]{geometry}
\usepackage{appendix}
\usepackage{colortbl}
\usepackage{physics}
\usepackage{float}
\usepackage{datetime}
\usepackage{hyperref}

\title{Praca Domowa Termodynamika i Fizyka Statystyczna R 2021/2022}
\author{Kacper Cybiński}
% \newdate{date}{28}{01}{2022}
% \date{\displaydate{date}}
\date{\today}
\setlength\parindent{0pt}

\newcommand{\com}[1]{{\color{red} #1}}

\newcommand{\link}[2]{{\color{cyan} \href{#1}{#2}}}

\renewcommand{\emph}{\textbf}

\begin{document}

\maketitle

\section{Zadanie 1}

Stefan i Janusz polują na kaczki, przy czym każdy z nich tak często trafia jak pudłuje. Stefan podczas polowania oddał po jednym strzale do 50 kaczek, a Janusz - do 51. Jakie jest prawdopodobieństwo, że Janusz ustrzelił więcej kaczek niż Stefan?

\section{Rozwiązanie}

Najwygodniej podejść do tego zadania zadając sobie pytanie o prowadzenie w przypadku gdy obydwaj oddali tyle samo strzałów. Weźmy sobie zdarzenie $A$, oznaczające że obaj ustrzelili tyle samo kaczek. Wtedy jego prawdopodobieństwo wynosi $p(A) = r$. Teraz racji tego, że prawdopodobieństwo każdego pojedyńczego trafienia to $p = \frac12$, równe dla obu panów, to prawdopodobieństwo zdarzenia $B$ - objęcia prowadzenia przez jednego z nich wyrazimy wzorem: 
\[
    p(B) = p \cdot p(A') = \frac12 \cdot (1-r) = \frac{1-r}{2}
\]
Przejdźmy teraz do sytuacji, gdy Janusz strzela 51 razy. Oznacza to, że ostatni strzał zrobi jakąkolwiek różnicę tylko i wyłącznie, gdy po 50 strzałach był remis, na co prawdopodobieństwo już sobie oznaczyliśmy wyżej jako $r$. Czyli sumując te dwa wydarzenia razem, szukane prawdopodobieństwo będzie wynosić:
\[
    p_s = \frac{1-r}{2} + \frac12 r = \frac12
\]
\end{document}