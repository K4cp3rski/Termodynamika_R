\documentclass[12pt,a4paper]{article}
\usepackage[utf8]{inputenc}
\usepackage[T1]{fontenc}
\usepackage{fontspec}
\usepackage[polish]{babel}
\usepackage{amsmath}
\usepackage{graphicx}
\usepackage[table,xcdraw]{xcolor}
\usepackage{hhline}
\usepackage{placeins}
\usepackage[margin=0.6in]{geometry}
\usepackage{appendix}
\usepackage{colortbl}
\usepackage{physics}
\usepackage{float}
\usepackage{datetime}
\usepackage{hyperref}

\title{Praca Domowa Termodynamika i Fizyka Statystyczna R 2021/2022}
\author{Kacper Cybiński}
% \newdate{date}{28}{01}{2022}
% \date{\displaydate{date}}
\date{\today}
\setlength\parindent{0pt}

\newcommand{\com}[1]{{\color{red} #1}}

\newcommand{\link}[2]{{\color{cyan} \href{#1}{#2}}}

\renewcommand{\emph}{\textbf}

\begin{document}

\maketitle

\section{Zadanie 2}


Na niekórych wsiach kurpiowskich zachował się następujący sposób przepowiadania czasu zam¡ąpójścia. Dziewczyna trzyma w dłoni sześć długich źdźbeł trawy, których końce wystają powyżej i poniżej dłoni. Druga dziewczyna wiąże górne końce parami, a trzecia - wi¡że parami dolne końce. Jeśli tak poa¡czone źdźbła trawy tworzą pierścień, to jest to niechybny sygnał, że pierwsza z dziewczyn w ciągu roku wyjdzie za mąż. Jakie jest prawdopodobieństwo, że pierścień zostanie utworzony, jeśli przyjmiemy, że źdźbła wiązane są w sposób przypadkowy?


\section{Rozwiązanie}

W tym przypadku można na to zadanie spojdzeć od strony kombinatorycznej. Mamy sześć źdźbeł trawy, każde o 6 końcach na górze i 6 na dole. Kolejność połączeń nie ma znaczenia, tj. połączenie z końców $1 \to 2$ jest równoważne połączeniu $2\to 1$. W związku z tym wiążąc dolne końce mamy:
\begin{itemize}
    \item $5$ możliwości wyboru pierwszego połączenia 
    \item $3$ możliwośći wyboru drugiego połączenia
    \item $1$ możliwość wyboru trzeciego połączenia
\end{itemize}
Dla końców górnych mamy:
\begin{itemize}
    \item $4$ możliwości wyboru pierwszego połączenia (jedna opcja nam odpada, tj ta, gdzie stworzylimyśmy tym połączeniem pierścień o długości niemaksymalnej.
    \item $2$ możliwośći wyboru drugiego połączenia, argument taki jak poprzednio.
    \item $1$ możliwość wyboru trzeciego połączenia
\end{itemize}
Łącznie dla zdarzenia $A$ - utworzenia pierścienia o maksymalnej długości daje nam to liczbę kombinacji $$A = 5 \cdot 3 \cdot 1 \cdot 4 \cdot 2 \cdot 1 = 120$$

Aby teraz wyznaczyć liczbę wszystkich możliwych kombinacji, to zdejmujemy ograniczenia z połączeń górnych, wtedy dostajemy $$\Omega = (5 \cdot 3 \cdot 1)^2 = 225$$

W związku z tym szukane prawdopodobieństwo to
\[
    p(A) = \frac{A}{\Omega} = \frac{120}{225} = \frac{8}{15}
\]
\end{document}