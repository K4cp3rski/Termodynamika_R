\documentclass[12pt,a4paper]{article}
\usepackage[utf8]{inputenc}
\usepackage[T1]{fontenc}
\usepackage{fontspec}
\usepackage[polish]{babel}
\usepackage{amsmath}
\usepackage{graphicx}
\usepackage[table,xcdraw]{xcolor}
\usepackage{hhline}
\usepackage{placeins}
\usepackage[margin=0.6in]{geometry}
\usepackage{appendix}
\usepackage{colortbl}
\usepackage{physics}
\usepackage{float}
\usepackage{datetime}
\usepackage{hyperref}

\title{Praca Domowa Termodynamika i Fizyka Statystyczna R 2021/2022}
\author{Kacper Cybiński}
% \newdate{date}{28}{01}{2022}
% \date{\displaydate{date}}
\date{\today}
\setlength\parindent{0pt}

\newcommand{\com}[1]{{\color{red} #1}}

\newcommand{\link}[2]{{\color{cyan} \href{#1}{#2}}}

\renewcommand{\emph}{\textbf}

\begin{document}

\maketitle

\section{Zadanie 3}

Rozważ reakcję zachodzącą pomiędzy gazem protonów (p), elektronów (e) i atomów wodoru (H):
$$
p+e \longleftrightarrow H
$$
Reakcja ta zachodzi w objętości $V \mathrm{w}$ kontakcie z termostatem o temperaturze $T$. Załóżmy też, że gazy można traktować jako gazy doskonałe nieoddziałujących cząstek, biorąc tylko pod uwagę degenerację stanów energetycznych ze względu na spin. Korzystając z zespołu wielkiego kanonicznego znajdź stężenie elektronów $n_{e}$ jako funkcję stężenia atomów wodoru $n_{H}$ i temperatury $T$ zakładając obojętność elektryczną układu. Dla uproszczenia załóż, że atomy wodoru mogą być tylko w stanie podstawowym o energii $-E_{0}$. Dodatkowo, w przypadku atomów wodoru zaniedbaj masę elektronu w porównaniu do masy protonu.

\section{Rozwiązanie}

Mamy w jednym pojemniku gaz protonowy, elektronowy i wodorowy. Jako że nie oddziahują one ze sobą (poza reakcją), to możemy je potraktować jako 3 osobne uklady. Dodatkowo potraktujmy reakcję $p+e \leftrightarrow H$ jako wymianę cząstek między tymi gazami, to znaczy przejście dodatkowej cząstki do gazu wodorowego jest równoważne zniknięciu po jednej cząstce z gazu protonowego i elektronowego i vice versa. Skoro gazy nie oddziałuja ze sobą, to wielka suma statystyczna ukladu jest równa:
$$
\Xi_{t o t}=\Xi_{e} \Xi_{p} \Xi_{H}
$$
Skoro w gazach stany energetyczne sia zdegenerowane, to gazy możemy potraktować, jak gdyby byly klasyczne. Policzmy wielkie sumy kanoniczne:
$$
\Xi_{e}=\sum_{N} z_{e}^{N} \int d \Gamma_{N} \exp \left(-\beta \frac{p^{2}}{2 m_{e}}\right)=\sum_{N} \frac{z_{e}^{N} V^{N}}{h^{3 N} N !}\left(2 \pi m_{e} k T\right)^{\frac{3}{2}}=\exp \left(\frac{z_{e} V}{h^{3}}\left(2 \pi m_{e} k T\right)^{\frac{3}{2}}\right)
$$
gdzie $z_{e}=\exp \left(\mu_{e} \beta\right)$ Dla uproszczenia zauważmy, że wyrażenie $\frac{h}{\sqrt{2 \pi m_{e} k T}}$ jest równe termicznej dlugości fali de Broglie'a $\lambda_{e}$. Czyli wielka suma kanoniczna dla gazu elektronowego jest równa:
$$
\Xi_{c}=\exp \left(\frac{z_{e} V}{\lambda_{e}^{3}}\right)
$$
Analogicznie dla gazu protonowego mamy:
$$
\Xi_{p}=\exp \left(\frac{z_{p} V}{\lambda_{p}^{3}}\right)
$$
Natomiast dla atomu wodoru mamy:
$$
\Xi_{H}=\sum_{N} z_{H}^{N} \int d \Gamma_{N} \exp \left(\beta E_{0}-\beta \frac{p^{2}}{2 m_{p}}\right)
$$
Czyli postępująe analogicznie jak weześniej dostajemy:
$$
\Xi_{H}=\exp \left(\frac{z_{H} V \exp \left(\beta E_{0}\right)}{\lambda_{p}^{3}}\right)
$$
we wzorze pojawilo się $\lambda_{p}$ zamiast $\lambda_{H}$, ponieważ pomijamy masę elektronu w atomie wodoru, więc dhugość termicznej fali de Broglie'a jest taka sama jak dla protonu. Stąd wielka suma calego ukladu:
$$
\Xi_{t o t}=\exp \left(\frac{z_{H} V \exp \left(\beta E_{0}\right)}{\lambda_{p}^{3}}\right) \exp \left(\frac{z_{p} V}{\lambda_{p}^{3}}\right) \exp \left(\frac{z_{c} V}{\lambda_{\varepsilon}^{3}}\right)
$$
Skoro układ jest elektrycznie obojętny to stężenie protonów i elektronów musi być równe. Dodatkowo skoro uklad jest w równowadze termodynamicznej ( $\mathrm{dS}=0, \mathrm{dE}=0 \mathrm{i} \mathrm{dV}=0$ ), to z równania:
$$
T d S=d E+p d V-\mu_{e} d N_{e}-\mu_{p} d N_{p}-\mu_{H} d N_{H}
$$

dostajeny, że
$$
\mu_{H} d N_{H}=-\mu_{c} d N_{e}-\mu_{p} d N_{p}
$$
ale skoro reakcja ma postac $p+e \leftrightarrow H$, to na pojawienie sie jednego atomu wodoru przypada znikniecie jednego elektronu i jednego protonu. Czyli mamy $d N_{h}=-d N_{e}=$ $-d N_{p}$. Korzystając z tej obserwacji, dostajemy zależnośc na potencjaly chemiczne gazów:
$$
\mu_{h}=\mu_{e}+\mu_{p}
$$
Policzmy teraz wielki potencjal:
$$
\Omega=-k T \log \Xi_{t o t}=-k T\left(\frac{z_{e} V}{\lambda_{e}^{3}}+\frac{z_{p} V}{\lambda_{p}^{3}}+\frac{z_{H} \exp \left(\beta E_{0}\right) V}{\lambda_{p}^{3}}\right)
$$
Policzmy teraz średnią liczbę elektronów:
$$
\begin{array}{r}
\left\langle N_{e}\right\rangle=-\frac{\partial \Omega}{\partial \mu_{e}} \\
\left\langle N_{e}\right\rangle=\frac{z_{e} V}{\lambda_{e}^{3}}
\end{array}
$$
Czyli stçżnie elektronów jest równe:
$$
n_{e}=\frac{z_{e}}{\lambda_{e}^{3}}
$$
Postepując analogicznie dostajemy, że stę́enje atomów wodoru jest równe:
$$
n_{H}=\frac{z_{H} \exp \left(\beta E_{0}\right)}{3}
$$
Chcemy teraz przeksztaleić wzór na stężenie elektronów, tak by stal się funkcją stężenia atomów wodoru i temperatury.
$$
\begin{gathered}
n_{e}=\frac{\exp \left(\mu_{e} \beta\right)}{\lambda_{e}^{3}}=\frac{\exp \left(\beta\left(\mu_{H}-\mu_{p}\right)\right)}{\lambda_{e}^{3}}=\frac{\exp \left(\mu_{h} \beta\right) \exp \left(\beta E_{0}\right) \exp \left(-\beta E_{0}\right) \lambda_{p}^{3}}{\exp \left(\mu_{p} \beta\right) \lambda_{e}^{3} \lambda_{p}^{3}} \\
n_{e}=\frac{n_{H} \exp \left(-\beta E_{0}\right)}{\lambda_{e}^{3} n_{p}}
\end{gathered}
$$
Ale wiemy, że $n_{e}=n_{p}$. Więc rơwnanie przybiera postać:
$$
n_{e}=\frac{n_{H} \exp \left(-\beta E_{0}\right)}{n_{e} \lambda_{e}^{3}}
$$
A stąd:
$$
n_{e}=\frac{\sqrt{n_{H}} \exp \left(-\frac{\beta E_{0}}{2}\right)}{\lambda_{e}^{\frac{3}{2}}}
$$


\end{document}