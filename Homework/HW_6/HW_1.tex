\documentclass[12pt,a4paper]{article}
\usepackage[utf8]{inputenc}
\usepackage[T1]{fontenc}
\usepackage{fontspec}
\usepackage[polish]{babel}
\usepackage{amsmath}
\usepackage{graphicx}
\usepackage[table,xcdraw]{xcolor}
\usepackage{hhline}
\usepackage{placeins}
\usepackage[margin=0.6in]{geometry}
\usepackage{appendix}
\usepackage{colortbl}
\usepackage{physics}
\usepackage{float}
\usepackage{datetime}
\usepackage{hyperref}

\title{Praca Domowa Termodynamika i Fizyka Statystyczna R 2021/2022}
\author{Kacper Cybiński}
% \newdate{date}{28}{01}{2022}
% \date{\displaydate{date}}
\date{\today}
\setlength\parindent{0pt}

\newcommand{\com}[1]{{\color{red} #1}}

\newcommand{\link}[2]{{\color{cyan} \href{#1}{#2}}}

\renewcommand{\emph}{\textbf}

\begin{document}

\maketitle

\section{Zadanie 1}

Rozważmy układ o ustalonej objętości V i temperaturze T. Pokaż, że fluktuacje energii w zespole wielkim kanonicznym i kanonicznym $(\Delta E)^2)$ różnią się o wyraz proporcjonalny do 
$(\Delta N)^2$. Jaka jest interpretacja tego wyniku?

\section{Rozwiązanie}

Rozważmy uklad o objętości $\mathrm{V}$ i temperaturze $\mathrm{T}$. Cheemy obliczyć różnicę:
$$
\left(\Delta E_{\omega k}\right)^{2}-\left(\Delta E_{k}\right)^{2}
$$
gdzie $\left(\Delta E_{w k}\right)^{2}$ to fluktuacje energii w zespole wielkokanonicznym, a $\left(\Delta E_{k}\right)^{2}$ to fluktuacje energii w zespole kanonicznym. Zacznijmy od fluktuacji w zespole kanonicznym. Srednia wartośc energii jest równa:
$$
\langle E\rangle=\frac{1}{Z} \sum_{i} E_{i} \exp \left(-\beta E_{i}\right)=-\left(\frac{\partial \log Z}{\partial \beta}\right)_{V, N}
$$
Natomiast wartość $\left\langle E^{2}\right\rangle$ jest równa:
$$
\begin{gathered}
\left\langle E^{2}\right\rangle=\frac{1}{Z} \sum_{i} E_{i}^{2} \exp \left(-\beta E_{i}\right)=-\frac{1}{Z} \frac{\partial}{\partial \beta}\left(\sum_{i} E_{i} \exp \left(-\beta E_{i}\right)\right) \\
\left\langle E^{2}\right\rangle=\frac{1}{Z} \frac{\partial^{2}}{\partial \beta^{2}}\left(\sum_{i} \exp \left(-\beta E_{i}\right)\right) \\
\left\langle E^{2}\right\rangle=\frac{\partial^{2} \log Z}{\partial \beta^{2}}+\frac{1}{Z^{2}}\left(\sum_{i} E_{i} \exp \left(-\beta E_{i}\right)\right)^{2}=\frac{\partial^{2} \log Z}{\partial \beta^{2}}+\langle E\rangle^{2}
\end{gathered}
$$
Czyli wartose fluktuacji jest równa:
$$
\left(\Delta E_{k}\right)^{2}=\left\langle E^{2}\right\rangle-\langle E\rangle^{2}=\frac{\partial^{2} \log Z}{\partial \beta^{2}}==k T^{2}\left(\frac{\partial\langle E\rangle}{\partial T}\right)_{V, N}
$$
gdzie $U$ to energia wewnętrzna ukladu. Przejdźmy teraz do zespołu wielkokanonicznego.
Srednia wartość energii jest równa:
$$
\langle E\rangle=\frac{1}{\Xi} \sum_{i} E_{i} \exp \left(-\beta E_{i}\right) z^{N}=-\left(\frac{\partial \log \Xi}{\partial \beta}\right)_{z, V}
$$
gdzie $z=\exp (\beta \mu)$. Tutaj zacząłem oznaczać, co jest stałe, ponieważ jest to ważne przy wyprowadzeniu. Policzmy teraz $\left\langle E^{2}\right\rangle$ :
$$
\begin{gathered}
\left\langle E^{2}\right\rangle=\frac{1}{\Xi} \sum_{i} E_{i}^{2} \exp \left(-\beta E_{i}\right) z^{N}=-\frac{1}{\Xi}\left(\frac{\partial}{\partial \beta} \sum E_{i} \exp \left(-\beta E_{i}\right) z^{N}\right)_{z, V} \\
\left\langle E^{2}\right\rangle=-\left(\frac{\partial\langle E\rangle}{\partial \beta}\right)_{z, V}+\frac{1}{\Xi^{2}}\left(\frac{\partial E}{\partial \beta}\right)_{z, V}^{2}
\end{gathered}
$$

Czyli $\left(\Delta E_{w k}\right)^{2}$ jest równe:
$$
\left(\Delta E_{w k}\right)^{2}=-\left(\frac{\partial(E)}{\partial \beta}\right)_{z, V}=k T^{2}\left(\frac{\partial\langle E\rangle}{\partial T}\right)_{z, V}
$$
Zauważmy teraz, ze w ukladzie wielkokanonicznym liczba caastek nie jest stala i także zalezy od temperatury. Skoro $\langle E\rangle$ jest funkcją liczby cząstek, to fluktuacje możemy zapisać jako:
$$
\left(\Delta E_{w, k}\right)^{2}=k T^{2}\left(\left(\frac{\partial\langle E\rangle}{\partial T}\right)_{N, V}+\left(\frac{\partial\langle E\rangle}{\partial\langle N\rangle}\right)_{T, V}\left(\frac{\partial\langle N\rangle}{\partial T}\right)_{z, V}\right)
$$
Rozpisując wzory na wartosć oczekiwanal liczby cząstek i energii zauważyć można, żee zachodzi:
$$
\left(\frac{\partial\langle N\rangle}{\partial T}\right)_{z, V}=\frac{1}{T}\left(\frac{\partial\langle E\rangle}{\partial \mu}\right)_{T, V}
$$
$Z$ drugiej strony mamy:
$$
\left(\frac{\partial\langle E\rangle}{\partial \mu}\right)_{T, V}=\left(\frac{\partial(E\rangle}{\partial\langle N\rangle}\right)_{T, V}\left(\frac{\partial\langle N\rangle}{\partial \mu}\right)_{T, V}
$$
Dodatkowo zauważmy, zee
$$
(\Delta N)^{2}=k T\left(\frac{\partial\langle N\rangle}{\partial \mu}\right)_{T, V}
$$
Stąd fluktuacje energii dla ukladu wielkokanonicanego będą postaci:
$$
\left(\Delta E_{w k}\right)^{2}=\left(\Delta E_{k}\right)^{2}+\left(\frac{\partial\langle E\rangle}{\partial\langle N\rangle}\right)_{T, V}^{2}(\Delta N)^{2}
$$
A zatem rôżnica fluktuacji będzie rôwna:
$$
\left(\Delta E_{w k}\right)^{2}-\left(\Delta E_{k}\right)^{2}=\left(\frac{\partial\langle E\rangle}{\partial\langle N\rangle}\right)_{T, V}^{2}(\Delta N)^{2}
$$
Czyli jest proporcjonalna do fluktuacji energii co należalo pokazać.


\end{document}