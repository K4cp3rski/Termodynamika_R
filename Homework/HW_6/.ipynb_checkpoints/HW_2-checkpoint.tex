\documentclass[12pt,a4paper]{article}
\usepackage[utf8]{inputenc}
\usepackage[T1]{fontenc}
\usepackage{fontspec}
\usepackage[polish]{babel}
\usepackage{amsmath}
\usepackage{graphicx}
\usepackage[table,xcdraw]{xcolor}
\usepackage{hhline}
\usepackage{placeins}
\usepackage[margin=0.6in]{geometry}
\usepackage{appendix}
\usepackage{colortbl}
\usepackage{physics}
\usepackage{float}
\usepackage{datetime}
\usepackage{hyperref}

\title{Praca Domowa Termodynamika i Fizyka Statystyczna R 2021/2022}
\author{Kacper Cybiński}
% \newdate{date}{28}{01}{2022}
% \date{\displaydate{date}}
\date{\today}
\setlength\parindent{0pt}

\newcommand{\com}[1]{{\color{red} #1}}

\newcommand{\link}[2]{{\color{cyan} \href{#1}{#2}}}

\renewcommand{\emph}{\textbf}

\begin{document}

\maketitle

\section{Zadanie 2}

W pudełku o objetości $V$ znajduje się ultrarelatywistyczny klasyczny gaz, dla którego $E=\sum_{i} c\left|\vec{p}_{i}\right|$. Oblicz $\Omega(\epsilon)=\frac{2 V}{h^{3}} \int_{\epsilon_{p}<\epsilon} d^{3} p$ opisujące liczbę stanów o energii mniejszej lub równej $\epsilon$. Znajdź wielką sumę statystyczną układu, jego energię swobodną i ciśnienie.
Wskazówka: sume po pędach mozna zastapić catka $\sum_{p} \longrightarrow \int_{0}^{\infty} d \epsilon \rho(\epsilon)$, gdzie $\rho=\Omega^{\prime}$

\section{Rozwiązanie}

Energia ultrarelatywistycznego gazu jest dana wzorem:
$$
E=\sum_{i} c\left|\vec{p}_{i}\right|
$$
Policzmy najpierw liczbę stanów o energii mniejszej od $\epsilon$. Liczba stanów dana jest wzorem:
$$
\Omega(\epsilon)=\frac{2 V}{h^{3}} \int_{\epsilon_{p}<e} d^{3} p
$$
W celu policzenia tej całki najłatwiej jest przejść do współrzędnych sferycznych gdzie $|\vec{p}|=r=\frac{\varepsilon_{p}}{c}$. Wówczas liczba stanów energii jest równa:
$$
\Omega(\epsilon)=\frac{2 V}{h^{3}} \int_{0}^{\frac{3}{2}} r^{2} \cos \theta d r d \phi d \theta=\frac{8 \pi V \epsilon^{3}}{3 h^{3} c^{3}}
$$
Znajdźmy teraz wielką sumę statystyczną układu (kladziemy $\exp (\beta \mu)=z)$ :
\begin{align*}
\Xi &=\sum_{N} z^{N} \int d \Gamma_{N}=\sum_{N} z^{N} \frac{V^{N}}{h^{3 N} N !}\int _ { 0 } ^ { \infty } d ^ { 3 } p \operatorname{exp} (-\beta). \\
\Xi &=\sum_{N} \frac{1}{N !}\left(\frac{8 \pi V z}{h^{3} c^{3} \beta^{3}}\right)^{N}=\exp \left(\frac{8 \pi V z}{h^{3} c^{3} \beta^{3}}\right)
\end{align*}
Stąd potencjał wielkokanoniczny jest równy:
$$
\eta=-k T \log \Xi=-\frac{8 \pi V z}{h^{3} c^{3} \beta^{4}}
$$
Z drugiej strony jednak:
$$
F-\mu N=\eta
$$
gdzie $F$ to szukana energia swobodna, a $N$ to średnia liczba cząstek. A zatem energia swobodna jest równa:
$$
F=-\frac{8 \pi V z}{h^{3} c^{3} \beta^{4}}+\mu N
$$
Natomiast ciśnienie jest równe:
$$
p=-\frac{\partial \eta}{\partial V}=\frac{8 \pi z}{h^{3} c^{3} \beta^{4}}
$$


\end{document}