\documentclass[12pt,a4paper]{article}
\usepackage[utf8]{inputenc}
\usepackage[T1]{fontenc}
\usepackage{fontspec}
\usepackage[polish]{babel}
\usepackage{amsmath}
\usepackage{graphicx}
\usepackage[table,xcdraw]{xcolor}
\usepackage{hhline}
\usepackage{placeins}
\usepackage[margin=0.6in]{geometry}
\usepackage{appendix}
\usepackage{colortbl}
\usepackage{physics}
\usepackage{float}
\usepackage{datetime}
\usepackage{hyperref}
\usepackage{tikz}
\usepackage[shortlabels]{enumitem}
\usetikzlibrary{calc}

\title{Praca Domowa Termodynamika i Fizyka Statystyczna R 2021/2022}
\author{Kacper Cybiński}
% \newdate{date}{28}{01}{2022}
% \date{\displaydate{date}}
\date{\today}
\setlength\parindent{0pt}

\newcommand{\com}[1]{{\color{red} #1}}

\newcommand{\link}[2]{{\color{cyan} \href{#1}{#2}}}

\renewcommand{\emph}{\textbf}

\DeclareDocumentCommand{\hcancel}{mO{0pt}O{0pt}O{0pt}O{0pt}}{%
    \tikz[baseline=(tocancel.base)]{
        \node[inner sep=0pt,outer sep=0pt] (tocancel) {#1};
        \draw[red] ($(tocancel.south west)+(#2,#3)$) -- ($(tocancel.north east)+(#4,#5)$);
    }%
}%

\begin{document}

\maketitle

\section{Zadanie 2}

Wyznaczyć stosunek temperatur Fermiego gazu doskonałego elektronów i protonów wewnątrz gwiazdy składającej się całkowicie ze zjonizowanego wodoru.

\section{Rozwiązanie}

Mamy powiedziane, że gwiazda z treści zadania jest w pełni zjonizowana, więc przyjmujemy populacje elektronów i protonów za równe. Jedne i drugie są fermionami, więc patrzymy na problem gazu(ów) fotonów w 3D. Dla uproszczenia przybliżmy gwiazdę jako sześcian o boku długości $L$. Jest to bez straty ogólności, bo aplikując periodyczne warunki brzegowe moglibyśmy rozszerzyć nasze rozumowanie na dowolną bryłę.
Mówimy sobie, że w tym pudełku mamy zarówno $N$ elektronów jak i $N$ protonów.//
Teraz już po ustaleniu założeń, które tu przyjmujemy możemy sobie przypomnieć, że energia jednego stanu kwantowego to będzie:
\[
    E_n = \frac{\pi^2 \hbar^2}{2 m L^2}\abs{\vb{n}}^2    
\]
Zaznaczmy, że $\vb{n}$ to skrótowy zapis na $\vb{n} = (n_x, n_y, n_z)$, bo takie rozbicie jest niezbędne z racji pracy w 3 wymiarach.\\
Zdefiniujmy sobie zero energetyczne na energii stanu podstawowego. Takie działanie upraszcza nam rachunki, bo wtedy liczba stanów o energii mniejszej od Energii Fermiego $E_F$ to liczba stanów mających $\abs{\vb{n}}$ mniejsze niż $\abs{\vb{n_F}}$, które odpowiada $E_F$.\\
Szukana liczba stanów będzie odpowiadać wycinkowi kuli dla którego wszystkie trzy współrzędne są dodatnie, z uwzględnieniem dwóch możliwości spinu cząstki. Stanowi on $\frac18$ całej kuli, czyli ta liczba stanów to będzie:
\[
    N = 2 \cdot \frac18 \frac43 \pi \abs{\vb{n_F}}^3 = \frac13 n_F^3 \implies n_F = \qty(\frac{3N}{\pi})^{\frac13}
\] 
Wyrażenie $n_F$ w języku całkowitej liczby cząstek przyda nam się do policzenia Energii Fermiego, gdyż jak już wcześniej wspomniane, $n_F$ odpowiada Energii Fermiego $E_F$, więc:
\[
    E_F = \frac{2 \pi^2 \hbar^2}{L^2} n_F^2 = \frac{\hbar^2}{2m} \qty(\frac{3 \pi^2 N}{V^2})^{\frac23}    
\]

A jako, że Temperatura Fermiego to $T_F = \frac{E_F}{k}$, więc:
\[
    T_F = \frac{1}{k} \frac{\hbar^2}{2m} \qty(\frac{2 \pi^2 N}{V^2})^{\frac23}    
\]

Doceolowo jednak chcemy porównać temperatury Fermiego dla obu gazów, więc po zapisaniu ich stosunku dostajemy:
\[    
    \frac{T_F^{(prot.)}}{T_F^{(elektr.)}} = \frac{\frac{1}{m_p} \hcancel{$\frac{\hbar^2}{2k}  \qty(\frac{2 \pi^2 N}{V^2})^{\frac23}$}}{\frac{1}{m_e} \hcancel{$\frac{\hbar^2}{2k} \qty(\frac{2 \pi^2 N}{V^2})^{\frac23}$}} = \frac{m_e}{m_p}
\]

Mogliśmy tu skrócić objętości obu gazów ze sobą, ponieważ mają równą liczbę cząstek $N$ i tą samą gęstosć (oby dwa są idealne i pochodzą ze zjonizowania wodoru), więc mają równe objętości.

\end{document}