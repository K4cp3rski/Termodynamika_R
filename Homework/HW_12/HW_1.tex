\documentclass[12pt,a4paper]{article}
\usepackage[utf8]{inputenc}
\usepackage[T1]{fontenc}
\usepackage{fontspec}
\usepackage[polish]{babel}
\usepackage{amsmath}
\usepackage{graphicx}
\usepackage[table,xcdraw]{xcolor}
\usepackage{hhline}
\usepackage{placeins}
\usepackage[margin=0.6in]{geometry}
\usepackage{appendix}
\usepackage{colortbl}
\usepackage{physics}
\usepackage{float}
\usepackage{datetime}
\usepackage{hyperref}

\title{Praca Domowa Termodynamika i Fizyka Statystyczna R 2021/2022}
\author{Kacper Cybiński}
% \newdate{date}{28}{01}{2022}
% \date{\displaydate{date}}
\date{\today}
\setlength\parindent{0pt}

\newcommand{\com}[1]{{\color{red} #1}}

\newcommand{\link}[2]{{\color{cyan} \href{#1}{#2}}}

\renewcommand{\emph}{\textbf}

\begin{document}

\maketitle

\section{Zadanie 1}

Pokaż, że entropia kwantowego gazu doskonałego wyraża się wzorem:
\[
    S = - k_B \sum_k \qty[\ev{n_k} \log \ev{n_k} \mp \qty(1 \pm \ev{n_k}) \log\qty(1 \pm \ev{n_k})]    
\]

Gdzie górny znak to Bozony a dolny to Fermiony.

\section{Rozwiązanie}

Pierwsze, co nam się przyda do pokazania tego, to wyznaczenie średniej liczby cząstek obsadzajacej stan o energii $\epsilon_k$. Do tego potrzebujemy wielkokanonicznego potencjału $\Omega_k$, czyli również sumy wielkokanonicznej $Z_k$. Są one równe odpowiednio:
\[
    Z_k = \qty(1 \pm \exp\qty(- \beta \qty(\epsilon_k - \mu)))^{\pm 1} \qc \Omega_k = \mp kT \log\qty(1 \pm \exp\qty(-\beta(\epsilon_k - \mu)))    
\]

Czyli teraz wiedząc, że $\ev{n_k} = - \pdv{\Omega_k}{\mu}$:
\[
   \ev{n_k} = \frac{\exp\qty(- \beta \qty(\epsilon_k - \mu))}{1 \pm \exp\qty(- \beta \qty(\epsilon_k - \mu))} = \frac{1}{\exp\qty(\beta \qty(\epsilon_k - \mu)) \pm 1}
\]

Mamy już potencjal wielkokanoniczny jednego stanu energetycznego, więc potencjał wielkokanoniczny całego układu $\Omega$ to będzie suma po potencjałach wszystkich dostępnych $k$ stanów, tj. $\Omega = \sum_k \Omega_k$.\\
Chcemy z tego dostać entropię, więc stosujemy tutaj wiedzę, że dostaniemy ją po zróżniczkowaniu po $T$, tj. $S = - \pdv{\Omega}{T}$:
\[
    S = \sum_j \pm k \log\qty(1 \pm \exp\qty(- \beta \qty(\epsilon_k - \mu))) + \frac{k T}{k T^2} (\epsilon_j - \mu) \frac{1}{\exp\qty(\beta \qty(\epsilon_k - \mu)) \pm 1}  
\]

We wzorze który chcemy udowodnić Entropia jest wyrażona w jęzku $\ev{n_k}$, więc szukamy sposobów na przejście na ten język. Pierwsza taka tożsamość, jaką tu zastosujemy, to:
\[
    \log(1 \pm \exp(-\beta (\epsilon_j - \mu))) = - \log(1 \mp \ev{n_k})    
\]
Da nam to wtedy:
\[
    S = \sum_j \mp k \log(1 \mp \ev{n_k}) - \log\qty(\exp(- \beta (\epsilon_j - \mu))) \ev{n_k}    
\]
Czyli po małym przearanżowaniu dostajemy:
\[
    S = -k \sum_j \ev{n_k}\log(\ev{n_k}) \mp (1 \pm \ev{n_k}) \log(1 \pm \ev{n_k})    
\]
Co należało dowieść.

\end{document}