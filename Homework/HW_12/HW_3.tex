\documentclass[12pt,a4paper]{article}
\usepackage[utf8]{inputenc}
\usepackage[T1]{fontenc}
\usepackage{fontspec}
\usepackage[polish]{babel}
\usepackage{amsmath}
\usepackage{graphicx}
\usepackage[table,xcdraw]{xcolor}
\usepackage{hhline}
\usepackage{placeins}
\usepackage[margin=0.6in]{geometry}
\usepackage{appendix}
\usepackage{colortbl}
\usepackage{physics}
\usepackage{float}
\usepackage{datetime}
\usepackage{hyperref}
\usepackage{tikz}
\usepackage[shortlabels]{enumitem}
\usetikzlibrary{calc}

\title{Praca Domowa Termodynamika i Fizyka Statystyczna R 2021/2022}
\author{Kacper Cybiński}
% \newdate{date}{28}{01}{2022}
% \date{\displaydate{date}}
\date{\today}
\setlength\parindent{0pt}

\newcommand{\com}[1]{{\color{red} #1}}

\newcommand{\link}[2]{{\color{cyan} \href{#1}{#2}}}

\renewcommand{\emph}{\textbf}

\begin{document}

\maketitle

\section{Zadanie 3}

Rozważ dwuwymiarowy gaz złożony z $N$ nieoddziałujących ze sobą nierelatywistycznych fermionów o spinie $\frac12$ w temperaturze $T = 0$. Gaz wypełnia powierzchnię o polu $A$.
\begin{enumerate}[a)]
    \item Znajdź energię Fermiego $E_F$ dla tego gazu \label{pkt_A}
    \item Znajdź energię wewnętrzną na cząstkę $u = \frac{U}{N}$ dla tego gazu jako funkcję $E_F$ \label{pkt_B}
\end{enumerate}

\section{Rozwiązanie}

Z definicji wszystkie rozpatrywane cząstki będą miały energię mniejszą niż Energia Fermiego. Skoro o powierzchni którą zajmuje gaz nie wiemy nic więcej, więc możemy bez straty ogólności utożsamić ją z kwadratem o boku $L$ i powierzchni $A$. Jak wiemy energia $n$-tego stanu to:
\[
    E_n = \frac{\pi^2 \hbar^2}{2 m L^2} n^2    
\]

Zdefiniujmy sobie zero energetyczne na energii stanu podstawowego. Takie działanie upraszcza nam rachunki, bo wtedy liczba stanów o energii mniejszej od Energii Fermiego $E_F$ to liczba stanów mających $\abs{\vb{n}}$ mniejsze niż $\abs{\vb{n_F}}$, które odpowiada $E_F$, a w tym przypadku i jednocześnie oznacza nam wszystkie możliwe stany.\\
Szukana liczba stanów będzie odpowiadać wycinkowi koła dla którego oby dwie współrzędne są dodatnie, z uwzględnieniem dwóch możliwości spinu cząstki. Stanowi on $\frac14$ całego koła, czyli ta liczba stanów i jednocześnie liczba wszystkich cząstek to będzie:
\[
    N = 2 \cdot \frac14 \pi n_F^2 \implies n_F = \qty(\frac{2N}{\pi})^{\frac12}
\] 
Da nam to Energię Fermiego równą:
\[
    E_F = \frac{\pi^2 \hbar^2}{2 m L^2} \frac{2N}{\pi}    
\]
Dzięki temu, że patrząc na tę powierzchnię zobaczyliśmy kwadrat o boku $L$, więc $L^2 = A$, co pozwala nam zapisać Energię Fermiego jako:
\[
    E_F = \frac{\pi \hbar^2}{m} \frac{N}{A}    
\]

Daje nam to odpowiedź na punkt \ref{pkt_A}.

Teraz policzmy sobie energię wewnętrzną na cząstkę tego gazu. Uznajemy, że stanów jest dużo, więc BSO sumocałka wchodzi tu w postaci całki. \\
Pomocniczo sobie jeszcze zdefiniujmy $n_E$ czyli liczbę stanów o energii mniejszej od $E$ i jest ona dana wzorem:
\begin{equation}\label{eq:ne}
    n_E = \frac{mE A}{\pi \hbar^2}
\end{equation}

Czyli finalnie liczymy całkę:

\[
    U = \int_0^N E \dd{n_E} = E_F N - \int_0^{E_F} n_E \dd{E}
\]

Co po wstawieniu równania \eqref{eq:ne} daje nam równanie:
\[
    U = E_F N - \int_0^{E_F} \frac{m E A}{\pi \hbar^2} \dd{E} = E_F N - \frac{N}{2 E_F} E_F^2 = \frac12 N E_F    
\]

Ale jako, że szukamy energii wewnętrznej na cząstkę, to szukany wzór to:
\[
    u = \frac12 E_F    
\]

Co daje nam odpowiedź na pytanie \ref{pkt_B}

\end{document}